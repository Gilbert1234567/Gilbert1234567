\documentclass[12pt]{report}
\usepackage{amsmath}
\usepackage{graphicx}
\usepackage{cite}
\usepackage{hyperref}
\usepackage{lipsum}

\title{Coverage of Climate Change Issues in Kenyan Print Media: A Case of Daily Nation and Standard Newspapers}
\author{Gilbert Kipkech \\ Dibrugarh University}
\date{12/11/2024}

\begin{document}

\maketitle

\begin{abstract}
    Climate change is an issue of public interest and, given the adverse effects it brings, concerted efforts are imperative as various stakeholders work towards solutions. The media, with its power to set and build agendas, plays a critical role, especially in its coverage of climate change. This study sought to analyze the extent to which climate change stories are given prominence, the nature of the coverage, and the drivers of climate change stories in \textit{The Daily Nation} and \textit{The Standard} newspapers between 2018 and 2019. The study was anchored on the Agenda Setting Theory to examine the prominence accorded to climate change stories, the Agenda Building Theory to assess the influence of external drivers of coverage, and the Framing Theory to examine how climate change stories are framed. The research adopted a descriptive content analysis research design. The study developed analysis criteria and a code sheet for content analysis, as well as an interview guide for key informant interviews with media house editors. Interviews were conducted with one editor from \textit{The Daily Nation} and one from \textit{The Standard}. A total of 1,730 articles were retrieved, focusing on four key terms: climate change, global warming, floods, and drought. The study found that Kenyan print media does not prioritize climate change stories by placing them on front pages. Media houses focused more on adaptation stories, with rare coverage of mitigation efforts. The study concluded that climate change coverage lacks prominence, with framing predominantly centered around disaster narratives and the portrayal of victims. The study recommends strategic placement of climate change stories on key pages and increased focus on mitigation efforts.
\end{abstract}

\chapter{Introduction}
\section{Background}
    Climate change is one of the most pressing global issues of our time, affecting ecosystems, economies, and societies worldwide. In Kenya, the effects of climate change are evident through increasing incidences of droughts, floods, and extreme weather patterns. These challenges have prompted discussions on the role of the media in raising awareness and shaping public perception about climate change.

    The role of the media is crucial in disseminating information and influencing public discourse on environmental issues. This study seeks to analyze how two of Kenya's leading newspapers, \textit{The Daily Nation} and \textit{The Standard}, have covered climate change-related stories between 2018 and 2019.

\section{Problem Statement}
    While there is a growing body of literature on climate change communication, little research has focused specifically on how Kenyan print media portrays climate change. Understanding the media's approach to climate change is crucial for assessing how the issue is framed and prioritized in the public domain.

\section{Objectives of the Study}
    The primary objectives of the study are:
    \begin{enumerate}
        \item To analyze the prominence given to climate change stories in \textit{The Daily Nation} and \textit{The Standard}.
        \item To examine the nature of the climate change stories covered by these newspapers.
        \item To identify the key drivers behind the climate change coverage in these media houses.
    \end{enumerate}

\chapter{Literature Review}
\section{The Role of Media in Climate Change Communication}
    Research has demonstrated that media plays a pivotal role in shaping public understanding of climate change (Smith, 2019). According to the Agenda Setting Theory, the media does not tell people what to think but rather what to think about. Through its coverage, the media can direct public attention to certain issues, influencing the public agenda.

\section{Framing of Climate Change in the Media}
    Framing Theory suggests that the way information is presented in the media can significantly influence how the public interprets it. In the context of climate change, the media often frames the issue in terms of disaster, which may shape perceptions of climate change as an immediate and urgent crisis.

\section{Climate Change in Kenyan Media}
    Studies have shown that climate change coverage in Kenyan media is generally sparse and often linked to disaster reporting (Mungai, 2020). However, there is also evidence that some media outlets have begun to emphasize adaptation strategies, such as community resilience and sustainable agriculture.

\chapter{Methodology}
\section{Research Design}
    This study used a descriptive content analysis design to analyze how climate change is covered in \textit{The Daily Nation} and \textit{The Standard}. Content analysis allows for the systematic examination of media content and provides a quantitative approach to identifying trends and patterns in media coverage.

\section{Data Collection Methods}
    \subsection{Content Analysis}
        A total of 1,730 articles were retrieved from both newspapers, focusing on key terms such as "climate change," "global warming," "floods," and "drought." The articles were categorized based on their prominence, the nature of the coverage, and the framing used.

    \subsection{Interviews with Editors}
        In-depth interviews were conducted with one editor from each newspaper to understand the editorial decision-making process and the factors influencing climate change coverage. The interviews were semi-structured and guided by an interview protocol.

\section{Data Analysis}
    The content was analyzed using a code sheet developed specifically for this study. Articles were coded based on variables such as the placement of the story, the tone of the story, the presence of mitigation or adaptation narratives, and the framing of climate change.

\chapter{Findings}
\section{Prominence of Climate Change Stories}
    The study found that climate change stories were generally not given prominent placement on the front pages of \textit{The Daily Nation} or \textit{The Standard}. Most climate change articles appeared on inner pages or in sections related to disaster management.

\section{Nature of Climate Change Stories}
    The majority of the climate change stories focused on adaptation efforts, including topics such as drought relief, water conservation, and agricultural practices. Very few articles discussed mitigation strategies, such as carbon emission reduction or policy interventions.

\section{Framing of Climate Change Stories}
    The framing of climate change stories often centered around disaster and crisis narratives. Victims of floods and droughts were frequently portrayed as the main actors in these stories, with government officials and NGOs featured in response narratives.

\section{Drivers of Climate Change Coverage}
    The key drivers of climate change coverage in both newspapers were external actors such as NGOs, government agencies, and international organizations. Editorial teams in both newspapers acknowledged that external reports and press releases were significant influences on the stories they covered.

\chapter{Conclusion and Recommendations}
\section{Conclusion}
    The study concluded that climate change is not given sufficient prominence in Kenyan print media. The coverage mainly focuses on adaptation stories, while mitigation efforts receive minimal attention. The framing of climate change is largely disaster-oriented, with a focus on immediate impacts rather than long-term solutions.

\section{Recommendations}
    \begin{enumerate}
        \item Media houses should consider placing climate change stories on the front pages to increase visibility and public engagement.
        \item There is a need for more balanced coverage, with greater focus on both adaptation and mitigation efforts.
        \item Media houses should prioritize diverse perspectives in climate change coverage, including voices from local communities, policy experts, and environmental activists.
    \end{enumerate}

\chapter{References}
    \begin{enumerate}
        \item Mungai, E. W. (2020). Climate change and media coverage in Kenya. \textit{Journal of Environmental Communication}, 12(3), 45-67.
        \item Smith, J. (2019). The role of media in climate change awareness. \textit{Climate Change and Society}, 8(2), 125-139.
    \end{enumerate}

\end{document}